\chapter{Literature Review}

\section{Cooperativism and Co-ops}

\paragraph{} In order to define cooperativism, it may be best to start with what it is not. Cooperatives as a structure are not the same as the act of collaborating. In order for a collaboration to be cooperative in nature, that collaboration would need take place within a group that is democratically-run, typically practiced as one-member-one-vote, with the members collectively deciding how their resources will be used to benefit them \autocite[54]{ratner_cooperativism_2009}. In the case of game development, teams of developers can collaborate in order to produce a game but that collaboration would not be cooperative unless its results were jointed owned and managed collectively by all the members. As such, cooperativism can best be described as the organizational principle in which the results of a group's collaboration are jointly-owned and democratically-controlled by its members.

\paragraph{} Co-ops are enterprises that are organized to practice cooperativism. These types of institutions are owned by the people who work there who then manage it democratically. This structure allows them to be run for member-benefit as each member of the co-op has the ability to directly participate in the decision-making process. The International Cooperative Alliance has adopted seven principles by which cooperatives are to put the value of coopertivism to practice: voluntary and open membership, democratic member control, member economic participation, autonomy and independence; education, training and information; cooperation among cooperatives, and concern for community\nocite{international_cooperative_alliance_cooperative_nodate}.

\paragraph{Voluntary and Open Membership} Cooperative teams are open to all persons interested and willing to accept the responsibilities of membership. Members of cooperatives voluntarily give their skills and resources to the team in order to contribute to a mutual goal. Membership of cooperative teams is available to all without gender, racial, political, religious or social discrimination. Fraternal benefit societies are an exception to this as their membership often shares one of the previously mentioned characteristics \autocite[16]{boland_introduction_2017}.

\paragraph{Democratic Member Control} This refers to both the practice of management through one-member-one-vote and the collective ownership of the results of the teamwork. Decisions within these types of teams require a consensus since each member has a vote; the validation for choices made comes from the majority of the members of the team voting to affirm them. The methods by which cooperative teams institute these paradigms vary based on their size, resources, and needs. Larger cooperative teams, for example, may elect their own management to handle complexity or delegate some decision-making \autocite[21]{northcountry_cooperative_foundation_worker_2006}. 

\paragraph{Member Economic Participation} Since cooperatives are collectively-owned, each member contributes equitably and democratically controls the capital of their co-op. Member of co-ops often receive financial benefits such as payment for their contributions. They also control over how the capital is spent. Common uses for capital at a co-op include development, building reserves, and support for external activities. Co-ops being operated for member benefit should not be misconstrued to mean they should not seek a profit; the distribution of the profit is part of the benefits of participation \autocite[29]{boland_introduction_2017}. 

\paragraph{Autonomy and Independence} Co-ops are self-organized and controlled by their members; the decisions they make are the decisions of the co-op. When a co-op enters agreements, raises revenue from external sources or engage in other external relations, they do so on their own accord and at the consent of the membership. Relations with external groups should also help to maintain the autonomy of the co-op. 

\paragraph{Education, Training and Information} In order for member of a co-op to effectively contribute, they need to have knowledge of how to do so and how to operate within a cooperative team. Additionally, co-ops also educate the general public about the benefits of cooperativism.

\paragraph{Cooperation Among Cooperatives} In order to strengthen the cooperative community, co-ops seek to collaborate. By sharing resources and working together on projects, co-ops can build support networks between them that benefit all their members. This also includes forming federations of co-ops to facilitate this collaboration.

\paragraph{Concern for Community} Co-ops work within their communities to engage in and promote sustainable policy. As with the other principles, community engagement is at the discretion of the membership. 

\paragraph{} Cooperative enterprises come in varying forms as well as serve varying purposes. Co-ops can be focused on a need within their community such as childcare or housing, an industry such as the arts or agriculture, or provide more general services such as utilities or financial assistance. What distinguishes them from conventional firms is their ownership and management by their members. Cooperatives also often form federations in order to cooperate amongst one another and share resources. 

\paragraph{} This study focuses on teams as they would exist in a worker-owned cooperative. These are co-ops in which workers combine their skills and resources in order to gain steady employment and income. This study focuses on this type of co-op because game studios that are cooperatively owned would be considered worker co-ops. Teams of developers in these types of studios use their skills, such as design, art, or programming, in order to create employment and income for themselves by producing games in a democratic workspace. Like conventional game studios, they can either produce and sell their own games or offer development as a service.

\paragraph{} The consensus among the literature is that both cooperativism and co-ops are predicated on collective ownership and some form of democratic decision-making. Carl Ratner does not offer any explicit strategies for the implementation of these values in \textit{Cooperativism: A Social, Economic, and Political Alternative to Capitalism} but rather describes the characteristics of various stages of cooperative activity (54-60). By contrast, the Northcountry Cooperative Foundation's \textit{Worker Co-op Toolbox} and Michael Boland's \textit{An Introduction to Cooperation and Mutualism} discuss strategies for successfully starting a co-op, structuring co-ops and handling elected management. Both of the aforementioned texts make reference to the International Cooperative Alliance's seven cooperative principles.

\section{Historical Overview}

\paragraph{} Worker-owned cooperatives first came to public attention as part of the labor movement during the Industrial Revolution in Europe and the United States. The transitioning of the production towards manufacturing processes led to the decline in various job sectors. This resulted in large sections of the workforce to shift to wage labour which could be insecure especially during economic downturns. In response to these changes, workers started to organize businesses of their own and controlling them \autocite{adams_putting_1992}. In 1844, the Rochdale Society of Equitable Pioneers established the Rochdale Principles which are the basis for the seven principles of the modern cooperative movement \autocite{thompson_co-op_1994}.

\paragraph{} Co-ops are a prominent and growing part of the modern global economy. In 2018, five pilot co-ops were given a \$1 million grant by Google to support economic development of cooperatives in the digital economy through critical analysis and designing of open source tools \autocite{the_new_school_trebor_2018}. Roughly 12\% of the world population were cooperators at any of the roughly three million cooperatives worldwide in 2020 \autocite{world_cooperative_monitor_exploring_2020}. Examples of established co-ops today include the Mondragon Corportation in Spain, Suma Wholefoods in the United Kingdom, and Means LLC in the United States.

\paragraph{} The longevity of the cooperative movement can be partially explained by the benefits their members receive from them. One of these is their resiliency with a study of three year survival rates of businesses in France from 2013 finding that worker co-ops had an 80\% to 90\% rate while the overall survival rate was 66\% \autocite{kruse_relative_2013}. Another study of co-ops in France found that the pay ratio between the highest and lowest paid workers was 14\% lower in co-ops than in conventional firms reflecting their perk of having more equitable distribution of profit \autocite{magne_wage_2017}. Members of co-ops can also be more productive than workers employed at traditional firms with a 1995 study of co-ops in the timber industry in the United States finding that they were 6\% - 14\% more efficient \autocite{craig_participation_1995}. Members of co-ops also report being happier with their job with a study of home health aids in the United States finding higher rates of satisfaction among those who were members of a co-op \autocite{kruse_effects_2013}.

% \section{Current Models}

% \section{Analysis}

% \section{Scope, Assumptions and Limitations}