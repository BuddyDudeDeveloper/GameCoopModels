\chapter{Findings}

\section{Overview}

\paragraph{} Data used for analysis and iteration as required by CAR for this study was collected from the cooperative team through interviews, observations and anonymous surveys. The individual interviews were designed to get each member's prospective on the action and how it could be improved. The participating members also filled out anonymous surveys to provide additional feedback that might not have been exposed during the interviews due to the personal involvement of the researcher in the study. Observations from the researcher directly in regards to the implementation of the model also contributed to the findings of the study as they were also a direct participant. Insights and findings from the interviews and surveys were derived from their codification.

\paragraph{} The first overall finding of note was the eagerness and willingness of the participants to engage in cooperative game development. The researcher noted that at the beginning of the study, the participants expressed excitement about developing games on a team where no one individual had complete creative control. This was further reflected in the researcher's observations on the increased participation of the team members; all of the members frequently attended meetings, contributed to the design, and updated one another on progress. This was also validated by the interviews and surveys in which the participants stated they felt more willing to participate as a result of the team being cooperative.

\paragraph{}

\begin{table}[h!]
    \centering
    \resizebox{\textwidth}{!}{%
    \begin{tabular}{|c|l|l|}
    \hline
    \textbf{Themes} &
      \multicolumn{1}{c|}{\textbf{Subthemes}} &
      \multicolumn{1}{c|}{\textbf{Examples}} \\ \hline
    \textbf{Communication} &
      \begin{tabular}[c]{@{}l@{}}Frequent discussions\\ Progress updates\\ Open discussions\end{tabular} &
      \begin{tabular}[c]{@{}l@{}}"Having multiple lines of communication helped us keep each other up-to-date."\\ "I was happy to share my progress and get feedback from my peers even when I fell behind or made mistakes."\\ "I felt like I could be more honest because my ideas would be considered equally with everyone else's."\end{tabular} \\ \hline
    \textbf{Collaboration} &
      \begin{tabular}[c]{@{}l@{}}Unique designs\\ Problem-solving\end{tabular} &
      \begin{tabular}[c]{@{}l@{}}"It was really cool having everyone contribute to decisions; it made our designs more interesting."\\ "I like that the democratic team made everyone participate and brought out ideas I would have never thought of."\\ "Working on a democratic team made me feel more respected and made it easier to work with the others."\\ "The game never felt like it was one person's idea; we all chipped in and it showed."\end{tabular} \\ \hline
    \textbf{Synchronization} &
      \begin{tabular}[c]{@{}l@{}}Notetaking\\ Task lists\\ Teaching\end{tabular} &
      \begin{tabular}[c]{@{}l@{}}"We were able to get everyone on the same page by writing everything down."\\ "Coming up with tasks together made it easier to figure out what we needed to do."\\ "We were able to show each other how did we things and I learned a lot because of it."\\ "When I was struggling on my tasks, I came to other members of the team to show me how they would do them."\end{tabular} \\ \hline
    \textbf{Participation} &
      \begin{tabular}[c]{@{}l@{}}Empowerment\\ Respected\\ Investment\end{tabular} &
      \begin{tabular}[c]{@{}l@{}}"I'm a shy person but the democratic team made me feel more confident and willing to share my thoughts."\\ "Everyone was respectful of one another because everyone was involved in making decisions."\\ "This type of team made me feel more invested in participating because I was more involved in decisions."\end{tabular} \\ \hline
    \textbf{Creativity} &
      \begin{tabular}[c]{@{}l@{}}Synthesis\\ Brainstorming\\ Peer feedback\end{tabular} &
      \begin{tabular}[c]{@{}l@{}}"Being able to bounce ideas off of my peers as equals helped make my work better."\\ "Having everyone involved in brainstorming let us take pieces of ideas and combine them into something original."\\ "When I got critique, it never felt like I was being 'talked down to' because we were all bosses."\end{tabular} \\ \hline
    \end{tabular}%
    }
    \caption{The codification of the final set of surveys and interviews.}
    \label{tab:my-table}
    \end{table}

\paragraph{} Another finding from this study was that the participants reported better communication as a result of using a cooperative model of production. The participants commented in interviews and surveys that they were more willing to share ideas and contribute to conversations because there was no singular authority deciding on the game. This was also observed by the researcher who noted that each member did frequently update one another on their progress and share their thoughts during meetings. Additionally, the participants also described feeling more respected and included because the production was jointly managed; this made them more inclined to communicate openly as a result. They also reported that they were less afraid to fail, be wrong or disagreed with because the structure of the team did not punish them for those things but rather sought to address them for the benefit of all the members.

\paragraph{} This study also found that the cooperative team was able to effectively leverage the collaborative aspects of cooperativism to benefit the project. The participants noted in interview and surveys that the designs for the game were a synthesis of everyone's ideas rather than the work of individual designers. They attributed this to their increased willingness to participate as a result of the inclusively reported previously by being apart of the management. The researcher also observed this during the meetings in which design was discussed as all the participants, including the researcher, all worked together to produce a unique level which took advantage of mechanics ideated by everyone on the team.

\paragraph{} An unexpected situation that occurred during the study was the ease of transitioning the study team to a democratic structure. The first iteration of the model was designed around when decisions needed to be made and when to align members. However, as the study began, the team quickly picked up on working cooperatively and making decisions democratic. This insight demonstrated that the problem was not when to be democratic during game development but how to effectively practice democratic game development effectively. This was further supported by the research done on problems experienced by modern game co-ops which emphasized issues of communication over voting. The researcher observed this towards the beginning of the study and it was subsequently confirmed through the surveys and interviews.

\paragraph{} The data and subsequent analysis are reliable and valid as they used CAR's rigorous methodology to collect data directly from participants. The members of the cooperative team provided their feedback on the model through interviews and anonymous surveys. This data was then used to inform the iterations of the model they then applied. The researcher was also a direct participant and their observations were used as data as well. The validity and reliability of this study are derived from the researcher's direct involvement in the action taking place and the subsequent data collected from those who made a game using the model.

\section{Analysis}

\paragraph{} Based on the data from the study, the model was successfully able to apply the principles of cooperativism to game development to make it viable for production. Although the first iteration of the model did not fully reflect the values of the theory and only addressed when decisions on a cooperative game development team should take place, the iterations that came after were effective at leveraging the advantages that come with this structure of team. This was validated by the feedback from the participants who expressed how cooperativism benefitted their work. The benefits reported by the participants, such as increased willingness to engage and more fruitful collaboration, reflected the research done on the benefits members of game development co-ops also experienced.

\paragraph{} Communication was one of the primary social problems in cooperative game development that the model was able to successfully address. The research found that the ability to make decisions and align members on a democratic team is often predicated on the team's ability to quickly share information and engage in fruitful discussion. The model was able to facilitate this because the cooperative aspect of the team's management encouraged the members to be more active with one another. This was channeled into the creation of artifacts that could communicate progress, such as text updates and progress boards, as well as into physical actions of communication such as meetings and small group sessions.

\paragraph{} The model was also successfully able to align members on the status of the game and facilitate the sharing of information. Participants were successfully able to build a shared understanding of the game and the goals of production as reflected in the data from the study because they were frequently in communication with one another. As previously stated, this communication itself was successful because cooperativism empowered the developers to do so effectively and productively. They were also able to effectively educate and share information with one another because, as the participants noted, having a lack of information or needing to be educated did not reflect negatively on the member but rather was a considered component of the model. The model instantiated this through open and inclusive communication and collaboration.

\paragraph{} The success of the communication and synchronization aspects of the model also fueled the success of the its collaboration aspect. This is due to the inclusive nature of the cooperative team; the environment of mutual understanding through open communication created the conditions necessary for the members to engage in work together productively. This allowed the developers to design aspects of the game together and create systems that were a synthesis of ideas. This was encouraged by the synchronicity of the team because alignment allowed the members to operate on the same mental model was they were working together.

\paragraph{} This study concludes that although the claim in the hypothesis was validated, the centering of production around democratic decision-making involves a more in-depth understanding of the conditions necessary to make meaningful decisions than when those decisions need to be made. As noted in the research, cooperative game development teams do not often have issues determining when key decisions needed to be made because it was clear during production when those needed to take place such as on what tools to use, which mechanics to add, and what changes needed to be made before specified deadlines. What was necessary for cooperativism to succeed in game development was creating an underlying apparatus that could facilitate the communication, alignment and collaboration needed to make those decisions. This was accomplished by using the principles of cooperativism to create an environment where members were empowered with implementable actions.

\paragraph{} This study has implications that should be considered for future cooperative game development and related research. The reported increase in willingness to participate as well as feelings of mutual respect and inclusiveness demonstrates not only the viability of cooperative game development broadly but also shows that it can benefit the production of games. Further research into long-term cooperative game development should be considered in order to investigate how these benefits can be translated into a larger scale production. This study, due to its limitations, could only focus on cooperative teams that did not have elected management; this is an area that could be explored with further research.