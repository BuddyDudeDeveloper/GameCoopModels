\chapter{Conclusion}

\section{Summary of the Study}

\paragraph{} This study examined the implications of applying the principles of cooperativism to game development in order to create a production model viable for game co-ops or other cooperative teams. This was accomplished by creating, implementing and iterating upon the model with a group of students working on a game for their course project over the course of six weeks using the Canonical Action Research (CAR) methodology. The study began with a broad examination of cooperativism as a theory, the history of the cooperative economy and game development, and an examination of the social problems that underpin traditional and cooperative game development. The study was then exited with an analysis of the results and the implications of them.

\paragraph{} The study found that cooperativism could not only be successfully applied to game development but that it could also benefit the developers engaging in it. The participants reported being more willing to engage with the team because of the respect and inclusiveness they experienced as part of the cooperative structure. They also reported having more fruitful collaborations sessions because members were able to openly express their thoughts and communicate issues they were having. This communication was also benefitted because of that same openness; the developers could share their progress or thoughts with one another in an environment where they were doing so as equal contributors working towards the same goals.

\section{Summary of the Literature Review}

\paragraph{} Cooperativism can best be described as the organizational principle in which the results of a group's collaboration are jointly-owned and democratically-controlled by its members. Every member of a co-op has the ability to contribute to the decision-making process because they are an owner and a participant. Cooperativism is based upon seven principles that seek to establish an environment conducive for co-ops to under their operations for member benefit. Co-ops may come in a variety of different forms and may exist to address different needs but all of them are based upon the previously mentioned seven principles.

\paragraph{} Co-ops as institutions have been apart of the world economy since the Industrial Revolution. They continue to be part of the economy today in the form of businesses, utilities and other types of enterprises. The longevity of the cooperative movement can be attributed to their demonstrated resiliency during economic hardship as well as the increased levels of happiness and productivity reported by studies on the members. Co-ops in the game industry are worker-owned cooperatives which means the studios in which the developers work at are also owned by them. Members of game co-ops confirmed the aforementioned benefits of cooperativism in addition to noting the necessity of communication for this type of structure to succeed.

\paragraph{} The most common problems experienced by game developers when creating games on teams tend to be social in nature rather than technical. Developers may need to find ways to work within the limitations of the tools which requires a degree of negotiation between it and the developer. Additionally, developers may also have to share knowledge with one another which can be dependant on both the ability of the educator to communicate their tacit knowledge and the mental model of the educatee. The heterogenous engineering necessary for successful game development can as a result be dependent on the social skills of the developers. These problems hold true for cooperative game development teams as well where socialization is a vital component to the successful implementation of democratic management.

\section{Summary of the Implications of the Study}

\paragraph{} The game development model created and iterated upon was able to address the previously mentioned social problems through the use of the principles of cooperativism. The involvement of all the members in the decision-making process allowed them to identify the limitations of their tools and decide as a group how to address them. Members were also able to better share and learn from the knowledge of others because cooperativism created an environment that supported it through the mutual respect assured by joint-management. When the issues were addressed, the model allowed the heterogenous engineering necessary for the successful production of a game on a cooperative team to occur.

\paragraph{} The model itself also represented a successful implementation of cooperative principles to game development. Exchanging information to align members on the status of development through communication reflects the principles of member participation and control. The outcome of educating, training and exchanging of information which are core principles of cooperativism and are embodied through synchronization. The collaboration component was reflective of member participation and control principles of cooperativism as the inclusion of the developers in the decision-making process is used to encourage further participation in creative, technical or other types of problem-solving activities.

\paragraph{} Future cooperative game development and related research may want to consider the implications of this study. The benefits of cooperativism described by the research and the feedback of the participants could be further explored as a tool for facilitating creativity in other domains. This study was unable to take into account long-term cooperative game development processes and as such there is opportunity for further exploration there. This is also true for researching the role elected management can play in continuing to facilitate cooperative game development. While not the conventional mode of production in the present time, working democratic and cooperatively could be a reasonable path forward for future game developers with tangible benefits to their productions.