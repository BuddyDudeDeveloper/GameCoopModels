\chapter{Introduction}

% Turns pages numbers back on.
\pagenumbering{arabic}

\section{Introduction to the Study}

% What is this study about? 
% Studio Studies / Ethnography

% Take out all references to unelected management
\paragraph{} Game design and development can be a creative and collaborative process for teams of developers with diverse sets of skills and experiences. However, game studios' traditional structure has often been hierarchical where the ownership has the authority to make all the production decisions. Worker-owned co-ops, by contrast, share the aforementioned characteristics of game development in the structure of their enterprise by being jointly owned and democratically controlled by the people who work there. To help facilitate success game development under this production paradigm, a new model is needed that considers the problems unique to developing a game in groups that make decisions collectively through democratic methodologies such as one-member-one-vote.

\paragraph{} This study seeks to understand cooperative game development challenges to create a model for game development co-ops that can facilitate consensus-building and creative collaboration within democratically-run teams. In order to simulate this type of group, a small team of students, alongside the researcher, applied this model to the development of their course project game for approximately six weeks. Their feedback, alongside the research contained within, was used to iterate upon the model to improve the processes and strategies for facilitating cooperative development. Further implications for the model and the future of cooperative game development were also discussed. 

\section{Defining the Problem}

% Traditional firms vs co-ops
% Social problems in game development and in democratic workspaces

\paragraph{} The day-to-day operations of game development often present problems in which the resolution depends on the developers' social skills. For example, developers may need to educate each other on work traditionally reserved for their role within a production to align the team on parts of the pipeline such as source control or batch asset exportation. Additionally, the education of that work or other work developers may engage in depends on the ability of more experienced developers to communicate their tacit knowledge of that subject. Collaboration with machines can also often be a problem where developers may need to circumvent the limitations of their tools to achieve their goals. Solutions to these problems often require social problem-solving and interpersonal communication skills rather than technical expertise \autocite{whitson_what_2020}.

\paragraph{} This study addresses the social problems within game development as they exist within cooperative teams using cooperativism as a framework for the solution. Developers engaging in democratic forms of production have to utilize similar social skills used to solve the aforementioned problems to share responsibility and foster an environment that facilitates collaboration and consensus-building. This is necessary to address the issues that arise from the interpersonal problems of collective management and game development. As such, there are opportunities for new models of game development that can address the social problems that arise from developing games in democratic teams.

\section{Reasons for Study}

\paragraph{} The goal of this study is to create a viable game development model for cooperatively-owned game studios to benefit existing ones or to encourage the growth of new ones. In 2020, approximately 12\% of the world population were cooperators at any of the roughly three million cooperatives worldwide \autocite{world_cooperative_monitor_exploring_2020}. Although they are not the predominant form of enterprise currently, cooperatives represent a sizable portion of the world economy. By developing a model for game development that could be implemented at a co-op, this study seeks to contribute to the expansion of that economic sector by providing a viable methodology for game developers to engage with this form of enterprise more effectively.

% Reframe cooperatives as a way to allow 
% Developers have hands-on experience developing game
% Decisions made by people who may not be a part of production or have experience with it
\paragraph{} Another goal of this study is to introduce a viable form of democratic production to the game industry. Developers are often the people within a studio with the most experience and knowledge of game development as they are the ones engaging in its day-to-day production. However, decisions within a traditionally structured game studio are often made independently of these developers. By creating a model for cooperative game development, developers can be empowered to manage the production of the games and apply their experience and knowledge to the direction of the end product.

\section{Context within Game Development}

\paragraph{} Cooperatives that currently exist in the game industry take a variety of forms. The most prominent form is a studio where the developers collectively create their own games to sell; examples of this include Motion-Twin and Studio Black Flag, both located in Bordeaux, France. Other studios, such as TESA Collective in the Massachusets and Talespinners in South Wales, opt instead to work with clients to produce games or components of them. There are also cooperative incubators such as Dutch Game Garden in the Netherlands and GamePlus in Australia which provide resources for developers engaging in this form of production.

\section{Hypothesis}

\paragraph{} Cooperativism can be applied to game development to create a viable production model for cooperatively-owned game studios by structuring the processes and procedures around democratic decision-making.