\chapter{Introduction}

% Turns pages numbers back on.
\pagenumbering{arabic}

\section{Introduction to the Study}

% What is this study about? 
% Studio Studies / Ethnography

\paragraph{} Game design and development can be a creative, collaborative and democratic process for teams of developers with diverse sets of skills and experiences. However, the traditional structure of game studios have often been hierarchical where the ownership or unelected management have the authority to make all the decisions in regards to production. Worker-owned cooperatives (co-ops), by contrast, share the aforementioned characteristics of game development in the structure of their enterprise by being jointly-owned and democratically-controlled by the people who work there. While successful game development methodologies exist, such as AGILE or SCRUM, these models do not take into account the problems unique to developing a game in a democratic workspace.

\paragraph{} This study seeks to understand the challenges of cooperative game development in order to create a model for game development co-ops that can facilitate consensus-building and creative collaboration within democratic workspaces. A small team of students applied the model to the development of their course project game over the course of the quarter for approximately seven weeks. Their feedback, alongside the research contained within, was used to iterate upon the model. Further implications for the model and the future of cooperative game development were also discussed. 

\section{Defining the Problem}

% Traditional firms vs co-ops
% Social problems in game development and in democratic workspaces

\paragraph{} The day-to-day operations of game development often present problems in which the resolution is dependent upon social skills of the developers. For example, developers may need to educate each other on work traditional reserved for their role within a production in order to align the team on parts of the pipeline such as source control or batch asset exportation. Additionally, the education of that work or of other work developers may engage in is dependent upon the ability of more experienced developers to communicate their tacit knowledge of that subject. Collaboration with machines can also often be a problem where developers may need to circumvent the limitations of their tools in order to achieve their goals. Solutions to these types of problems often require social problem-solving and interpersonal communication skills rather than technical expertise \autocite{whitson_what_2020}.

\paragraph{} This study addresses the social problems within game development as they exist within cooperative workspaces. Developers engaging in democratic forms of production have to utilize similar social skills used to solve the aforementioned problems to share responsibility and foster an environment that facilitates collaboration and consensus-building. Current game development methodologies such as AGILE or SCRUM address problems with project organization and timelines by providing rituals and artifacts to help developers and firms navigate the process \autocite{keith_agile_2020}. These models do not take into account the aforementioned social problems or those present in cooperative workspaces; as such, there are opportunities for new models of game development that can address the social problems that arise from developing games in democratic teams.

\section{Reasons for Study}

\paragraph{} The goal of this study is to create a viable game development model for cooperatively-owned game studios to benefit existing ones or to encourage the growth of new ones. In 2020, approximately 12\% of the world population are cooperators at any of the roughly there million cooperatives worldwide \autocite{world_cooperative_monitor_exploring_2020}. Although they are not the predominant form of enterprise currently, cooperatives represent a sizable portion of the world economy. By developing a model for game development that could be implemented at a co-op, this study seeks to contribute to the expansion of that economic sector by providing a viable methodology for game developers to engage with this form of enterprise more effectively.

\paragraph{} Another goal of this study is to introduce a viable form of democratic production to the game industry which has historically been predominantly autocratic in management. While small teams of developers may collaborate with one another through egalitarian methods, the structures of firms within the game industry are often heirarchical with decisions being made by unelected administrations. While not always problematic in execution, this type of management system has historically resulted in the exploitation of developers with the most notable example of which being extended work hours towards a deadline typically referred to as "crunch" \autocite{woodcock_for_2020}. Introducing a feasible game development model for co-ops may incentivize developers who are interested in pursuing their own enterprises to engage in this form of business as a form of protection from exploitation by unelected management.

\section{Context within Game Development}

\paragraph{} Cooperatives that currently exist in the game industry take a variety of forms. The most prominent form is a studio where the developers collectively create their own games to sell; examples of this include MotionTwin and Studio Black Flag. Other studios, such as TESA Collective and Talespinners, opt instead to work with clients to produce games or components of them. There are also cooperative incubators such as Dutch Game Garden and GamePlus which provide resources for developers engaging in this form of production.

\section{Hypothesis}

Cooperativism can be applied to game development to create a viable model for cooperatively-owned game studios by structuring the processes and procedures around democratic decision-making.